\documentclass[12pt]{article}
 \usepackage[margin=1in]{geometry} 
\usepackage{amsmath,amsthm,amssymb,amsfonts}
\usepackage[brazilian]{babel}
\usepackage[utf8]{inputenc}
\usepackage[T1]{fontenc}
 \DeclareMathOperator{\diam}{diam}
 \DeclareMathOperator{\viz}{viz}
 \DeclareMathOperator{\tq}{t.q.}
 \DeclareMathOperator{\Rn}{\mathbb{R}^n}
 \newcommand{\norm}[1]{\left\lVert#1\right\rVert}
 \newcommand\restrict[1]{\raisebox{-.5ex}{$|$}_{#1}}
 
\newcommand{\N}{\mathbb{N}}
\newcommand{\Z}{\mathbb{Z}}
 
\newenvironment{problem}[2][Problem]{\begin{trivlist}
\item[\hskip \labelsep {\bfseries #1}\hskip \labelsep {\bfseries #2.}]}{\end{trivlist}}
%If you want to title your bold things something different just make another thing exactly like this but replace "problem" with the name of the thing you want, like theorem or lemma or whatever
 
\begin{document}
 
 \title{Derivação do método de integração que aproxima pelo quadrado}
 \author{Danilo de Freitas Naiff}
 \date{}
 \maketitle
 Idéia (\textit{Sampling for Inference in Probabilistic Models with Fast Bayesian Quadrature}):
 Temos uma integral $\mathbb{E}[ l(X)] = \int l(x) p(x) dx$, que queremos aproximar, 
 \textit{sabendo que $l$ é uma verossimilhança}. Assumimos $p(x)$ Gaussiana  
 $\mathcal{N}(x;\nu,\Lambda)$ (em geral multivariada). Aproximamos então $\tilde{l}(x) = \sqrt{2 l(x)}$ 
 (no paper original tem um parametro $\alpha$ próximo de zero, mas tiramos por enquanto) por um processo Gaussiano. Logo $l(x) = \frac{1}{2} \tilde{l}(x)^2$. Temos então, que dado $x_\mathcal{D}$ pontos 
 onde $\tilde{l}$ é medido, temos que
 \begin{equation}
 \tilde{m}_\mathcal{D}(x) = \mathbf{k}^T (x,x_\mathcal{D}) K_\mathcal{D}^{-1} \tilde{l}(x_\mathcal{D})
 \end{equation}. 
 Fazendo uma linearização (um pouco roubada), temos que
 \begin{equation}
 l(x) \approx \frac{1}{2} \tilde{m}_\mathcal{D}(x)^2, 
 \end{equation}
 De onde podemos integrar. Talvez seja mais fácil pensar na aproximação acima como uma 
 regressão da verossimilhança, que preserva:
 \begin{itemize}
 	\item Positividade
 	\item Extrapolação para zero fora da região de amostragem.
 \end{itemize}
 Adaptamos essa idéia para a aproximação da marginalização de um Processo Gaussiano, inspirado 
 fortemente em \textit{Gaussian Processes for Prediction}.
 
 Temos em geral:
 \begin{equation}\label{eq:marg}
	p(y|\mathcal{D},I) = \frac{1}{Z} \int p(y | \phi, \mathcal{D},I) p(\phi| \mathcal{D},I) 
		p(\phi | I) d \phi
 \end{equation}
 A partir de agora tiramos a dependência de $I$ por conveniência. Sabemos que, dado $\phi$, 
 $p(y | \phi, \mathcal{D}) = f_{N(y | \mu_\phi, \Sigma_\phi)}(x)$. Então, dado $\phi_S$ 
 amostras dos hiperparâmetros, consideramos $l(\phi) = p(y | \phi, \mathcal{D},I) p(\phi| \mathcal{D},I)$. 
 Assim (e usamos aqui fortemente o \textit{Matrix Cookbook}):
 \begin{equation}
	\tilde{l}(\phi) = \sqrt{2} (4 \pi)^{n/4} |\Sigma_{\phi}|^{1/4} f_{N(\mu_\phi, 2 \Sigma_\phi)}(y) 
		\sqrt{a(\phi)}
 \end{equation}
 Onde $a(\phi) = p(\phi| \mathcal{D},I)$. Então, aproximamos \eqref{eq:marg} por:
 \begin{equation}\label{eq:approx1}
 \begin{split}
 p(y | \mathcal{D},\phi_s) \approx & \frac{1}{Z} \int \frac{1}{2} \tilde{m}_{\phi_s}(\phi) p(\phi) 
             d\phi \\
	 = & \frac{1}{2 Z} \int \tilde{l}(\phi_s)^T K_{\phi_s}^{-1} \mathbf{k}(\phi,\phi^s) 
						    \mathbf{k}(\phi,\phi^s)^T K_{\phi_s}^{-1} \tilde{l}(\phi_s) 
						    p(\phi) d\phi \\
	 = & \frac{(4 \pi)^{n/2}}{Z} \sum_{i,j} \Big( M_{i,j} |\Sigma_{(\phi_i)}|^{1/4} |\Sigma_{(\phi_j)}|^{1/4} 
			   \sqrt{a(\phi_i)} \sqrt{a(\phi_j)} \\
	 & f_{N(\mu_{\phi_i}, 2 \Sigma_{\phi_i})}(y) f_{N(\mu_{\phi_j}, 2 \Sigma_{\phi_j})}(y) \Big)
 \end{split}
 \end{equation}
 Onde:
 \begin{equation}
  \begin{split}
  & M = K_{\phi_s}^{-1} W K_{\phi_s}^{-1} \\
  & W_{i,j} = \int k(\phi,\phi_i) k(\phi,\phi_j) p(\phi) d \phi
  \end{split}
 \end{equation}
 Mas temos que:
 \begin{equation}
  \begin{split}
   & f_{N(\mu_{\phi_i}, 2 \Sigma_{\phi_i})}(y) f_{N(\mu_{\phi_j}, 2 \Sigma_{\phi_j})}(y) = 
    C(\phi_i,\phi_j) f_{N(\mu_{i,j}, \Sigma_{i,j})}(y) \\
    & C(\phi_i,\phi_j) = \frac{1}{(4 \pi)^{n/2} |\Sigma_{\phi_i} + \Sigma_{\phi_j}|^{1/2}} 
					         \exp \Big(-\frac{1}{4} (\mu_{\phi_i} - \mu_{\phi_j})^T (\Sigma_{\phi_i} + \Sigma_{\phi_j})^{-1} (\mu_{\phi_i} - \mu_{\phi_j})\Big) \\
	& \mu_{i,j} = (\Sigma_{\phi_i}^{-1} + \Sigma_{\phi_j}^{-1})^{-1} (\Sigma_{\phi_i}^{-1} \mu_{\phi_j} +
																		\Sigma_{\phi_j}^{-1} \mu_{\phi_i}) \\
	& \Sigma_{i,j} = 2 (\Sigma_{\phi_i}^{-1} + \Sigma_{\phi_j}^{-1})^{-1}
  \end{split}
 \end{equation}
 Juntando os termos constantes em uma nova constante de normalização (também chamada $Z$), temos então que:
 \begin{equation}
  p(y | \mathcal{D},\phi_s) \approx \frac{1}{Z} \sum_{i,j}  M_{i,j} |\Sigma_{(\phi_i)}|^{1/4} |\Sigma_{(\phi_j)}|^{1/4} 
  \sqrt{a(\phi_i) a(\phi_j)} C(\phi_i,\phi_j) f_{N(\mu_{i,j}, \Sigma_{i,j})}(y)
 \end{equation}
 Para calcular $W$, assumimos que $k(\phi_1,\phi_2)$ é um \textit{squared-exponential}, ou seja, 
 $k(\phi_1,\phi_2) = h^2 \exp(-\frac{1}{2} \sum_{k=1}^d (\phi_1^{(k)} - \phi_2^{(k)})^2/l_k^2)$, 
 e que $p(\phi) = f_{\mathcal{N}(\nu,{diag}(\lambda^2))}(\phi) = 
 \prod_{k=1}^d f_{N(\nu_k,\lambda_k^2}(\phi)$, temos que:
 \begin{equation}
  W_{i,j} = h^2 \prod_{k=1}^{d} \int_{-\infty}^{\infty}
	  \frac{1}{\sqrt{2 \pi \lambda_k^2}}\exp \Big(-\frac{1}{2} \Big( 
		  \frac{(\phi^{(k)} - \phi_j^{(k)})^2}{l_k^2} + \frac{(\phi^{(k)} - \phi_j^{(k)})^2}{l_k^2} 
		    + + \frac{(\phi^{(k)} - \nu_k)^2}{\lambda_k^2} \Big) \Big)
 \end{equation}
 Simplificando a integral acima, temos que, completando quadrado,
 \begin{equation}
 \begin{split}
  & W_{i,j} = h^2 \prod_{k=1}^{d} \Big( \frac{V_k^2}{\lambda_k^2} \Big)^{1/2} \exp \Big(-\frac{1}{2} C_k \Big) \\
  & V_k^2 = \left( \frac{2}{l_k^2} + \frac{1}{\lambda_k^2} \right)^{-1} \\
  & C_k = \frac{(\phi_i^{(k)})^2 + (\phi_j^{(k)})^2}{l_k^2} + \frac{\nu_k^2}{\lambda_k^2} 
    - \left(\frac{2}{l_k^2} + \frac{1}{\lambda_k^2} \right)^{-1} 
    \left(\frac{\phi_i^{(k)} + \phi_j^{(k)}}{l_k^2} + \frac{\nu_k}{\lambda_k^{2}} \right)^2
 \end{split}
 \end{equation}
\end{document}